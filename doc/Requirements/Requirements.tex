\documentclass{report}

\usepackage{graphicx}
\usepackage{fancyhdr}

\pagestyle{fancyplain}

\lhead[\fancyplain{}{\bfseries\thepage}]
        {\fancyplain{}{\bfseries\rightmark}}
\rhead[\fancyplain{}{\bfseries\leftmark}]
        {\fancyplain{}{\bfseries\thepage}}
\lfoot[]{\fancyplain{}{\bfseries\scriptsize YardSale: The Open
        Source Point of Sale Solution}} \cfoot{}

%left margin gutter
\oddsidemargin 0.0in
%right margin gutter
\evensidemargin 1.0in
\textwidth 6.5in
%distance from bottom of top margin to top of writing
\headheight 0.0in
\topmargin 0.0in
\textheight 8.5in

\begin{document}

\begin{titlepage}
    \vspace*{2cm}
    \begin{center}

        {\LARGE {\bf INTERIM REPORT}}\\
        \vspace*{1cm}

        \LARGE{ YardSale\\The Open Source Point of Sale Solution\\ }

        \includegraphics{aslogic_smaller.png}

        \large{{\bf A.S. Logic Systems Co.}\\Jesse Lovelace\\Adam
        Parrish\\Mike Swigon\\Jay Johnston\\John Lamb\\Cameron Watts\\}
        \vspace*{0.5cm}

        {April 7, 2004}
    \end{center}
\end{titlepage}

\tableofcontents

\chapter{Requirements Definition}

    \section{Introduction}

        YardSale is an open source Point of Sale system that is being
        developed by A.S Logic Systems.  Current
        implementations of Point of Sale systems are fraught with a number
        of issues including often being extremely overpriced, difficult
        to administer, dated in their functionality, and lacking necessary
        operations.

        The power of an open source system such as YardSale is the
        open way in which it is designed.  A user of the YardSale system
        can take off the shelf computer hardware (in many cases older hardware
        will perform just as well) and create a point of sale with minimal
        configuration.  In addition, supporting new hardware is trivial since
        the YardSale interface specification is in no way hindered by
        restrictive close-source licenses.

        The name "YardSale" was concieved to emphasize the
        versatility and simplicity of the system.  The components of a
        YardSale system can be older, used hardware and
        the underlying software can be completely free.

        Being that YardSale is an open source project the initial
        market will target small, locally owned retail
        stores.  This type of market will allow for extensive on-site
        research, as many of these businesses should be willing to work
        with us to improve on the functionality of their existing POS
        system.  In addition, because A.S. Logic has decided to take the
        open source route, the software itself will be freely available to
        all who wish to use it, the only expense that will come into play
        is if the company wishes to enlist our services in setup,
        troubleshooting, support, expansion, or customization.

        However, we believe that YardSale can be easily used by
        those who do not wish to purchase support contracts due to
        the clear documentation and well-written interfaces.

    \section{Functional Requirements}

        This section entails all of the minimal requirements set by the AS
        Logic Systems development team for the Open Source Point of Sale, hereafter referred
        to as YardSale.\\
        \\YardSale should accomplish the following Point of Sale operations:

        \begin{itemize}
            \item {Manage Point of Sale Related Data}
            \begin{itemize}
                \item {Inventory Data}
                \begin{itemize}
                    \item {Add a product to the inventory}
                    \item {Edit an existing inventory item}
                    \item {Search for an inventory item by various criteria}
                    \item {Remove an inventory item}
                    \item {Add an inventory item to a transaction}
                \end{itemize}
                \item{Customer Data}
                \begin{itemize}
                    \item{Add and remove customers}
                    \item{Search for customers by various criteria}
                    \item{Edit customer information}
                    \item{Associate a customer with a transaction}
                \end{itemize}
                \item {Employee Data}
                \begin{itemize}
                    \item{Add and remove employees}
                    \item{Associate an access level with each employee}
                    \item{Associate an employee with a
                    transaction}
                    \item{Associate Login information with each
                    employee}
                \end{itemize}
                \item{Transaction Data}
                \begin{itemize}
                    \item{Add new transaction information}
                    \item{Search for transactions by various
                    criteria}
                    \item{Associate values with a transaction}
                \end{itemize}
            \end{itemize}
            \item{Perform Point of Sale Transactions}
            \begin{itemize}
                \item{ Allow sale and return of inventory items between
                customers and employees }
                \item{ Calculate the tax on an item being sold }
                \item{ Record all transaction information }
            \end{itemize}
            \item{Provide Detailed Reporting}
            \begin{itemize}
                \item Allow a manager to produce reports
                related to the Point of Sale information
                recorded by the system.
            \end{itemize}
            \item{Allow for User Login with Differing Access Levels}
            \begin{itemize}
                \item Administrator (Super User) Access
                \item Manager Access
                \item Sales Associate Access
            \end{itemize}
        \end{itemize}


        Each of these tasks is described in-depth in the corresponding
        subsection. The functionality outlined above is designed to be as
        extensible as possible; with the correct implementation of these
        requirements, the YardSale System will support the expansion
        of functionality with great ease.

        \subsection{Information Management}
            YardSale will store information about all aspects of program operation
            data. The data that is stored can be broken down into the four following
            subsections:

            \begin{itemize}
                \item{Inventory Data}
                \item{Customer Data}
                \item{Employee Data}
                \item{Transaction Data}
            \end{itemize}

            \subsubsection{Inventory Data}
                YardSale will store data pertaining to all
                inventory items currently in-stock and on-order.
                The following fields will be recorded for each
                item:

                \begin{itemize}
                    \item SKU Number
                    \item Barcode Number
                    \item Item Name
                    \item Corresponding Department
                    \item Item Type
                    \item Retail Price
                    \item Wholesale Price
                    \item Weight (in pounds)
                    \item Associated Vendor
                    \item Associated Tax Type
                    \item Item Description
                    \item Shipping Flags
                    \begin{itemize}
                        \item Freight Only?
                        \item Oversized?
                    \end{itemize}
                    \item Number In-Stock
                    \item Number On-Order
                    \item Reorder Level
                \end{itemize}

                \noindent YardSale shall provide the following functionality
                to all inventory data in the system:

                \begin{itemize}
                    \item {\bf Add New Inventory:} new inventory items
                    shall be able to be added to the current
                    inventory.
                    \item {\bf Search Inventory:} all inventory
                    items shall be available to a search function;
                    items retrieved based on specified criteria.
                    \item {\bf Modify Inventory Item:} all inventory items
                    shall be able to have any field modified and
                    saved.
                    \item {\bf Remove Inventory:} existing inventory
                    items shall be able to be removed from the
                    list of current inventory items.
                    \item {\bf Associate Inventory Item with a Transaction:}
                    each inventory item shall be able to be
                    associated with a transaction in which the
                    item's inventory status was altered ({\sl see
                    TRANSACTION DATA}).
                \end{itemize}

            \subsubsection{Customer Data}
                YardSale will store data pertaining to all
                customers of the system's business.
                The following fields will be recorded for each
                customer:

                \begin{itemize}
                    \item Customer Account Number
                    \item First Name
                    \item Middle Name
                    \item Last Name
                    \item Address
                    \item Phone Number
                    \item City
                    \item State
                    \item Zip Code
                    \item Credit Card Number
                    \item Credit Card Expiration Date
                    \item Name on Credit Card
                    \item Signature
                    %\item Photo
                \end{itemize}

                \noindent YardSale shall provide the following functionality
                to all customer data in the system:

                \begin{itemize}
                    \item {\bf Add Customer:} new customers shall
                    be able to be added to the current customer
                    list.
                    \item {\bf Search Customers:} all customers
                     shall be available to a search function;
                    customers displayed based on specified criteria.
                    \item {\bf Modify Customer:} all customers
                    shall be able to have any field modified and
                    saved.
                    \item {\bf Associate Customer with a Transaction:}
                    customers shall be able to be
                    associated with a transaction in which they
                    altered the state of some inventory item ({\sl see
                    TRANSACTION DATA}).
                \end{itemize}

            \subsubsection{Employee Data}
                YardSale will store data pertaining to all
                employees currently employed by the business.
                The following fields will be recorded for each
                employee:

                \begin{itemize}
                    \item Social Security Number
                    \item Employee ID Number
                    \item First Name
                    \item Middle Name
                    \item Last Name
                    \item Address
                    \item Phone Number
                    \item City
                    \item State
                    \item Zip Code
                    %\item Picture
                    \item Signature
                    \item Access Control Level ({\sl see section
                    1.2.4})
                \end{itemize}

                \noindent YardSale shall provide the following functionality
                to all employee data in the system:

                \begin{itemize}
                    \item {\bf Add New Employee:} new employees
                    shall be able to be added to the current
                    employment list.
                    \item {\bf Remove Employee:} existing
                    employees shall be able to be removed from the
                    list of current employees.
                    \item {\bf Associate Employees with an Access Control Level:}
                    each existing employee shall be able to be
                    associated with a predefined access level to
                    the system.
                    \item {\bf Associate Employee with a Transaction:}
                    employees shall be able to be
                    associated with a transaction in which they
                    altered the state of some inventory item ({\sl see
                    TRANSACTION DATA}).
                \end{itemize}

                \noindent Related to Employee Information, the
                system shall record information pertaining to user
                logins.  This information should be inherent
                within the system and store the following
                information upon login and logout of each user:
                \begin{itemize}
                    \item Employee ID Number
                    \item Date and Time Logged In
                    \item Date and Time Logged Out
                \end{itemize}

            \subsubsection{Transaction Data}
                YardSale will store data pertaining to all
                transactions made by the system.
                The following fields will be recorded for each
                transaction:

                \begin{itemize}
                    \item Employee ID Number (external reference)
                    \item Inventory SKU Number (external
                    reference)
                    \item Customer Account Number (external
                    reference)
                    \item Transaction ID Number
                    \item Sale Price
                    \item Sale Quantity
                    \item Comments
                \end{itemize}

                \noindent YardSale shall provide the following functionality
                to all transaction data in the system:

                \begin{itemize}
                    \item {\bf Add New Transaction:} new
                    transaction data shall be able to be
                    added to the current transaction list.  This
                    should occur automatically with the completion
                    of any transaction.
                    \item {\bf Search Transactions:} all
                    transactions shall be available to a search function;
                    transactions displayed based on specified criteria.
                    \item {\bf Associate Values with a
                    Transaction:} externally referenced values;
                    such as employee ID, SKU, and customer
                    account numbers, shall be able to be
                    referenced and saved with each transaction.
                \end{itemize}


        \subsection{Point of Sale Transactions}
            YardSale shall allow for Point of Sale Transactions of inventory items
            to be made between employees and customers.  Each
            transaction shall satisfy the following criteria:

            \begin{itemize}
                \item Must contain one or more inventory items to
                be either sold or returned.
                \item Sales transaction functionality shall be available
                to all current employees in the system, regardless
                of access control level.
                \item Return transaction functionality shall only
                be available to employees with appropriately
                specified access control level.
                \item Must associate proper tax type/information
                with each item of a transaction.
                \item Completion of a transaction shall create new
                Transaction Data to be stored by the system;
                including the following external data:
                \begin{itemize}
                    \item {\bf Customer Information:} Transactions
                    shall be linked with an associated customer stored
                    in the system.
                    \item {\bf Employee Information:} Transactions
                    shall be linked with an associated employee stored
                    in the system.
                    \item {\bf Inventory Information:} all information
                    pertaining to the inventory items altered during
                    the transaction (either added or removed).
                \end{itemize}
                \noindent {\sl see TRANSACTION DATA of section 1.2.1}
            \end{itemize}


        \subsection{Reporting}
            YardSale shall provide detailed reporting mechanisms
            to managers of the system.  These reports shall be
            automatically generated by the system from the data stored
            by the Information Management System.  Reporting shall
            be available via an easily accessible template to
            users with the appropriate user access level.  The
            following reports shall be available by default:

            \begin{itemize}
                \item {\bf Hourly Reporting for Employees:} this
                report will provide an exact output of total hours
                each employee was logged into the system.
                \item {\bf Total Transactions by Time Period:} this report will
                provide an output of all transaction information
                stored during the specified time period.
                \item {\bf Sales Transactions by Employee:} this report
                will provide an output of all sales transactions
                made by the selected employee(s), or all
                employees.
            \end{itemize}

            \noindent Each of these reports shall be made
            available for printing in a commonly used format,
            supported by all operating systems.

        \subsection{Login/User Access Levels}
            YardSale shall allow for users to log into the system
            with predefined system access.  One of the following
            User Access Levels shall be associated with every user
            of the system:

            \subsubsection{Sales Associate Functionality}
                The Sales Associate User shall be the
                primary user of the YardSale system.  A Sales Associate
                must be a valid employee of the business currently
                running the system.  This access
                level is the lowest and most basic of the three
                User Access Levels, as its rights are given to all
                current employees by default.  Sales Associates
                shall be given access to the following areas of the
                system:

                \begin{itemize}
                    \item Customer Information Management
                    \item Sales/Return Transaction Processing
                \end{itemize}

            \subsubsection{Manager Functionality}
                The Manager User shall be given nearly full
                functionality to the YardSale system.  A Manager must be a
                valid employee of the business currently running the
                system.  This access level's functionality
                is designed to allow for maintenance of most
                areas of the system.  Upon login, the following options shall be
                available to a Manager:

                \begin{itemize}
                    \item Inventory Information Management
                    \item Customer Information Management
                    \item Employee Information Management
                    \item Reporting Mechanisms
                    \item Sales/Return Transaction Processing
                \end{itemize}

            \subsubsection{Administrative Functionality}
                The Administrative User shall be given full
                functionality of the YardSale system, and all
                related, underlying systems (Information Management
                System, Operating System...).  An Administrator
                need not be an employee of the business currently
                running the system.  This will allow for valid
                entry of technicians and A.S. Logic
                Systems employees for servicing and setup of the
                system.  The Administrative User is a "Super User"
                and should not be used in the day-to-day routines
                of the YardSale system.  Upon login, the following
                options shall be available to an Administrator:

                \begin{itemize}
                    \item Inventory Information Management
                    \item Customer Information Management
                    \item Employee Information Management
                    \item Reporting Mechanisms
                    \item Sales/Return Transaction Processing
                    \item System Configuration
                \end{itemize}

    \section{Non Functional Requirements}

        \subsection{Operating System Independent}
            The YardSale system shall be designed to be compatible
            with all Operating System platforms available in
            today's marketplace.

        \subsection{Architecture Independent}
            The YardSale system shall be designed to be
            independent of all architectural aspects of the
            underlying system.  This is the main aspect of
            YardSale, as it should be able to run appropriately on
            any system meeting the Operating System's Minimal
            Requirements.

    \section{System Constraints}
        \subsection{Interfaces}
            The YardSale system must support the use of several peripheral interfaces
            to aid in the completion of POS transactions and other functionality outlined
            previously.  The interfaces are outlined as follows:

            \begin{itemize}
                \item {\bf Cash Drawer:} The system must support the use of a
                cash drawer for storage of money exchanged during transactions.
                It shall pop open upon the completion of a transaction or when
                prompted by the user.

                \item {\bf Magnetic Card Scanner:} The system must support the use
                of a magnetic card scanner for automatic input of information.  These
                scanners must correctly input information stored on both credit cards
                and employee access cards.

                \item {\bf Barcode Scanner:} The system must support the use of
                a barcode scanner for automatic input of items for inventory
                and transaction processing.

                \item {\bf Receipt Printer:} The system must support the use of
                a receipt printer.  The printers must print specified information
                at the conclusion of each transaction and will also be used with
                the reporting functionality.

                \item {\bf TouchScreen Monitor:} The system must support the use
                of TouchScreen monitors.  The monitor shall act as both input and
                output devices for the system.  The monitors must correctly read
                the commands given and relay them to the system.
                In addition, the system's user interface must be
                easily used with the TouchScreen monitor.
            \end{itemize}

    \section{External Dependencies and Interfaces}
	    \subsection{Host Operating System}
	        The YardSale system requires the use of a host operating system
		to run the YardSale software.  The constraints on this operating
		system are minimal and nearly all currently used operating systems
		are supported by YardSale.  YardSale shall be designed to function
		properly with the minimal system requirements of the host
		operating system.

	    \subsection{Input Processing Device}
		In addition to the devices defined by the System Constraints
		({\sl see SYSTEM CONSTRAINTS sec 1.4.1}), some additional input
		devices are required for entering information into the system.
		Examples of such devices are a keyboard or mouse.

	    \subsection{Output Processing Device}
		In addition to the devices defined by the System Constraints
		({\sl see SYSTEM CONSTRAINTS sec 1.4.1}), some additional output	
		devices are required for retrieving information from the system.
		Examples of such devices are a monitor and printer.

    \section{Preliminary Design}

        \subsection{Major Modules}

            \subsubsection{Database}
                Information Management in YardSale will be handled
                through the use of a database.  The databse will
                utilize the querying power of mySQL4 incorporated
                with ODBC. \\
                The Database module works as the translator between
                the user and the database.  It converts calls
                made by the user into SQL queries to be sent to the
                database.  When the query results are returned from the
                database,the database converts these items to a
                Database Type ({\sl see DATABASE TYPES}) to be read by the
                user interface. OTL Libraries will be used in conjunction with ODBC to
                provide database connectivity with the user interface. OTL is
                a cross platform C++ library for ODBC.

            \subsubsection{Database Types}
                The Database Types module is the superclass for all information that
                may be sent to the user interface from the database.  It contains
                function calls for each of the following types:

                \begin{list}{}
                    \item{{\bf Inventory Information Type:} contains variables for all
                    fields defined by the Inventory Information Management section
                    of this document.  Each of these fields will be stored in the
                    Inventory Table of the database.  Coordinates with the Inventory Management
                    view of the user interface, which is also referenced by Transactions.}

                    \item{{\bf Customer Information Type:} contains variables for all
                    fields defined by the Customer Information Management section
                    of this document.  Each of these fields will be stored in the
                    Customer Table of the database.  Coordinates with the Customer Management
                    view of the user interface, which is also referenced by Transactions.}

                    \item{{\bf Employee Information Type:} contains variables for all
                    fields defined by the Employee Information Management section
                    of this document.  Each of these fields will be stored in the
                    Employee Table of the database.  Coordinates with the Employee Management
                    view of the user interface.}

                    \item{{\bf Transaction Information Type:} contains variables for all
                    fields defined by the Transaction Information Management section
                    of this document.  Each of these fields will be stored in the
                    Transaction Log Table of the database.
                    Information is automatically updated upon the
                    conclusion of a Transaction in the user
                    interface.}
                \end{list}{}

            \subsubsection{GUI wxWidges}
                The User Interface for YardSale will be designed
                using the wxWidges libraries for C++.\\
                The GUI wxWidges module defines all screens and windows used by the
                user interface.  It contains functions derived from the
                wxWidges libraries and specifies which functions to call
                for each action within the user interface.  It also interacts with the
                Database module to display information sent from the
                database.\\

%        \newpage


        \subsection{System Architecture}

            The figure shown below outlines the class dependencies and hierarchy of
            the modules described in section 1.5.1.\\

            \includegraphics{yardsale_modules.png}\\
            %\caption{System Architecture Flow Chart}

    \section{Preliminary Project Task Plan}
        %Scope and Size of Project Paragraph

        \subsection{Milestones}
            \begin{description}
                \item[2004-03-05] FEATURE FREEZE - deadline for adding functional
                features to the design of the POS.

                \item[2004-03-07] ITERATION 1 - all requirements for first
                iteration to be completed and documented in the interim
                report.

                \item[2004-04-01] ITERATION 2 - all secondary requirements
                to be completed for the second iteration.

                \item[2004-04-17] CODE FREEZE - deadline for beginning the
                coding of features; those not already implemented will be removed;
                testing only beyond this point

                \item[2004-04-27] FINAL DELIVERABLE - completion of
                prototype for presentation to the CSC department at the
                Posters and Pies event.
            \end{description}

        \subsection{Team Member Roles and Responsibilities}
            The following figure displays a graphical hierarchy of the team's
            roles:\\

            \includegraphics{organization.png}
            \\The team roles are defined as follows:

            \begin{description}

                \item[Jesse Lovelace] As the CEO of A.S. Logic Systems,
                Jesse is in charge of all aspects of the project.  Though
                he works very closely with both VPs, Jesse is the ultimate
                decision maker for the team.  In addition to his roles as
                CEO, Jesse is also in charge of the design and
                implementation of the User Interface and all security
                aspects.

                \item[Adam Parrish] Adam is the Vice President in charge
                of Development at A.S Logic Systems.  He works closely
                with both the CEO and VP of Marketing to see that
                YardSale's implementation is both correct and timely.  In
                addition to these company roles, Adam is also the lead
                Database Programmer; seeing to it that the database is
                functioning properly and creating the SQL scripts for
                populating and querying the database.

                \item[Mike Swigon] Mike is the Vice President of
                Marketing at A.S. Logic Systems.  He works closely with
                both Adam and Jesse to both design the entire system and
                to insure that its implementation is correct.  In addition
                to these responsibilities, Mike works with Adam in database
                setup and creation of SQL scripts for populating and
                querying the database.

                \item[John Lamb] John's responsibilities fall primarily in
                interfacing with the database.  He works closely with Jesse
                to develop a UI that can correctly and securely
                communicate information to and from the database; also
                implementing the SQL statements developed by Mike and
                Adam.

                \item[Jay Johnston] Jay's responsibilities fall primarily
                in creation of the UI and networking the system during
                setup.  He works closely with Jesse to develop modules
                for required functionality of the interface.

                \item[Cameron Watts] Cameron's responsibilities fall
                primarily on marketing and system research.  He works
                closely with the design group to ensure YardSale's
                functionality is top-of-the-line and user friendly.
            \end{description}

\chapter{Design}

    The following sections are related to the design of YardSale
    in regards to the database level design as well as the client
    application level design. A database was chosen to store the
    data for the application since we require persistent data
    storage. A database also is a robust method for storage, and
    easily queried as opposed to flat text files or binary file
    storage methods. There are also built in user access rights
    which will aid in security for YardSale.

    Our Chosen Relational Database Management System for development is
    MySQL 4 although since all of our queries are based on the
    SQL standard it should be easy to allow the use of any
    SQL Compliant RDBMS. For Application to Database connectivity
    we plan to use OTL in conjuction with ODBC. OTL is a set of C++
    libraries which provide a higher level SQL layer between the
    actual database and the application. ODBC is used to interface
    with the actual database on an operating system and network
    level.

    The following first section provides a high level
    explanation of the database architecture using UML diagrams. The
    latter section explains design related issues of the YardSale
    client program.

\section{YardSale Database Design}

    The YardSale database was designed with security and efficiency in
    mind. The goal was, as it is in any database, to seperate unrelated
    data and key similar items together with table relations. The global
    database model is seen in the diagram on the following page. Although
    it is not necessarily easy to read the table relations can be seen, and
    table specifics like field values, indexes, and foreign and primary keys
    can be seen in the following table diagrams.

    \newpage

    {\bf YardSale Global Database Diagram}\\
    \\
    \\
    %   \includegraphics{Database_Layout_REDUCED.png}

%    \includegraphics[scale=1]{Database_Layout_REDUCED.png}



    \begin{enumerate}
        \item{Customer Table}
        \begin{itemize}
            \item{Primary Key: Customer Account Number}
        \end{itemize}
        \item{Inventory Table}
        \begin{itemize}
            \item{Primary Key: Inventory SKU Number}
        \end{itemize}
        \item{Tax Table}
        \begin{itemize}
            \item{Primary Key: Tax ID}
        \end{itemize}
        \item{Transaction Log Table}
        \begin{itemize}
            \item{Primary Key: Key composed of all referenced keys}
        \end{itemize}
        \item{Vendor Table}
        \begin{itemize}
            \item{Primary Key: Vendor ID}
        \end{itemize}
        \item{Employee Table}
        \begin{itemize}
            \item{Primary Key: Employee Social Security Number}
        \end{itemize}
        \item{Package Table}
        \begin{itemize}
            \item{Primary Key: Package ID Number}
        \end{itemize}
        \item{Login Table}
        \begin{itemize}
            \item{Primary Key: Key composed of all referenced keys}
        \end{itemize}
        \item{ACL Table}
        \begin{itemize}
            \item{Primary Key: ACL Type}
        \end{itemize}
        \item{Carrier Table}
        \begin{itemize}
            \item{Primary Key: Carrier ID}
        \end{itemize}
        \item{Shipping Table}
        \begin{itemize}
            \item{Primary Key: Key composed of all referenced keys}
        \end{itemize}
        \item{Key Table}
        \begin{itemize}
            \item{Primary Key: Key ID}
        \end{itemize}
    \end{enumerate}

    \newpage

        \subsection{Customer Management}

        Since customer management is basically a simple task it is maintained in only
        one table. There is no real need to have their data stored accross many
        different tables since all of the information is just personal
        data.\\

        {\bf Customer Table}\\
        \\
        \\
%        \includegraphics{Tables/CustomerTable.png}\\
        \\
        The above UML diagram shows the layout of the customer table. The object of this
        table is to manage customer personal information for later reference in
        transactions. By allowing this associativity YardSale will later be able to
        formulate reports on how much each of its customers spend for example. It also
        provides a rather in depth directory of all of a business's clients for use
        in any form they see fit.

        The data stored begins with the customer account number which is just an
        arbitrarily assigned number that is managed by the database management
        system. It is also the primary key for the table and is therefore unique.
        First, Middle, and Last names are all stored in seperate fields of their own,
        as is the Address information. Credit information can also be stored about users
        but is optional, and is also planned to be stored in the field as an encrypted
        string of data so that underprivileged users can not access it. Two interesting
        features about this table is the ability to store a link to a photograph and
        signature for each customer. That way it will aid in positively identifying a
        customer when they are checking out with check or credit.


        \newpage

        \subsection{Inventory Management}

        YardSale's Inventory Management scheme spans over three tables. The main table that
        stores actual inventory description information is the Inventory Table itself. The
        other two tables are used to support the main table. They are the Tax table and the
        Vendor Table. The tax table is basically a storage area for different tax types, and
        the Vendor Table is a storage table for Vendor information or Inventory Supplier
        information.\\
        \\
        {\bf Inventory Table}\\
        \\
        \\
%        \includegraphics{Tables/InventoryTable.png}\\
        \\
        \\
        This table being the main Inventory storage structure contains all information needed
        about any inventory item. The record primary key is the SKU number which is a user
        defined number or character string. These are often used in small businesses to
        internally key their inventory. Manufacturers will often have a bar code associated
        with each item the produce as well so their is a field available for that as well.
        Each item can be briefly described, and associated with a department for further
        subcatagorizing. The number of any particular item is maintained as well as how many
        of the item are on order. There is a field for storing the number at which an item should
        be reordered, and also how many to reorder at that time. There are varying description
        fields such as item type and weight as well. Three pricing fields are supplied. The
        first two are statically maintained as retail and wholesale price. The third price
        type varies on the number of items being purchased. This field is maintained in XML
        format so that as many different pricing levels as are needed can be defined. When an
        item is received it updates the field corresponding to last received. Some items are
        oddly shaped or are overly heavy, either of these two options could cause the oversized
        flag to be set, and if the oversized flag is set a ship by freight option would also be
        set, but they are mutually exclusive and the ship by freight can be set without the
        oversized flag being set.\\
        \\
        {\bf Tax Table}\\
        \\
        \\
%        \includegraphics{Tables/TaxTable.png}\\
        \\
        \\
        The tax table is just a support table for the inventory items. It is referenced by ID
        depending on the desired taxing an item should have. Using a table allows for user definable
        tax types, and allows the database to be flexible to tax changes. All that the table needs
        is a Tax Name and a percentage for taxing items. The ID field is managed internally by the
        database management system and is also the primary key for the table.\\
        \\
        {\bf Vendor Table}\\
        \\
        \\
%        \includegraphics{Tables/VendorTable.png}\\
        \\
        \\
        The Vendor Table is used also as a supporting table for the inventory. This information
        pertains to the supplier of the items being sold. When an item reaches its reorder level
        in the Inventory Table, the information in this table would be used to make the order to
        resupply. The table is keyed by a unique ID that is managed by the database management
        system. The company name, address, and pertinent contact information is maintained along
        with a sales representative's name. There are optional fields for company specialty, email
        address, and homepage as well.

        \subsection{Transaction Handling}

        Transaction Handling is a process that has data spanning two tables with two more supporting
        tables. Transaction handling is split into two sections. The first section being the actual
        day to day transaction, and then the added functionality of packaging the items sold during
        the transaction.\\
        \\
        \\
        {\bf Transaction Table}\\
        \\
        \\
%        \includegraphics{Tables/TransactionLogTable.png}\\
        \\
        \\
        The Transaction Log Table is used to store information that links Customers, Employees, and
        inventory items. Each entry in the transaction table represents an item sold during a transaction.
        Since the key is not a single ID number it is the combination of the Customer, Employee, Item,
        Quantity and Price, the ID can be used to represent the overall transaction. Any row that contains
        an equivalent ID belongs to the same transaction.
        \\
        \\
        {\bf Package Table}\\
        \\
        \\
%        \includegraphics{Tables/PackageTable.png}\\
        \\
        \\
        The Package Table is used to store information about a package. There is a link to a transaction ID
        so that items can be associated with a package. There is also a link to a customer from the package
        table. The reason there is a link from the package table is because one customer may wish to buy
        something for another so the customer who made the transaction will not necessarily be the same as
        the one who will receive the package. There is also a reference to a carrier and a shipping type.
        A tracking number field is provided for when the tracking number is issued by the Carrier.
        \\
        \\
        {\bf Carrier Table}\\
        \\
        \\
%        \includegraphics{Tables/CarrierTable.png}\\
        \\
        \\
        The Carrier table is a support table for the Package table. It provides information about the different
        shipping services. The information maintained here is just what is necessary to get a package shipped.
        There is a phone number, pickup location and an ID associated with each entry.
        \\
        \\
        {\bf Shipping Table}\\
        \\
        \\
        The Shipping Table is also a support table for the Package Table. This table is a list of all of the
        different methods of shipping available associated with the Carrier Table. Since each Carrier can
        have multiple shipping methods they can all be easily added and deleted if they ever change via
        this table. The table just contains the name for the type of shipping, the carrier, and how much
        it costs.\\
        \\
%        \includegraphics{Tables/ShippingTable.png}\\
        \\
        \subsection{Employee Management}

        The Employee Management section is actually two sections. There is the Employee portion which
        spans two tables, the Employee Table for employee information, and the Login Table which
        keeps track of the hours an employee works between logging in and out of the clients. The
        other section is the security aspect of the program as it relates to employees. It also spans
        two tables; the ACL Table and the Key Table.\\
        \\
        {\bf Employee Table}\\
        \\
        \\
%        \includegraphics{Tables/EmployeeTable.png}
        \\
        \\
        The Employee Table keeps track of all pertinent information about employees that would
        be needed by an employer. Basic personal information such as First, Middle, and Last name
        as well as Contact information are maintained here. Each employee in the table has a unique
        employee identification number associated with them to avoid having to use the Social
        Security Number as a key. This also allows data such as the Social Security number to be
        encrypted to alleviate underprivileged users from seeing it. Each employee also has a
        field for a picture link and signature link to help with positive identification. The other
        field in this table is the password field which is used to store a users password to allow
        a user to login, and which is also used to decrypt the keys from the key table. Each user
        is also associated with a ACL entry in the ACL Table.\\
        \\
        {\bf Login Table}\\
        \\
        \\
%        \includegraphics{Tables/LoginTable.png}\\
        \\
        \\
        The Login Table tracks the amount of time an employee has been logged in based on when
        they logged into their first client to when they logout of their last client. The structure
        is very simple. It references an employee ID and has a Count field. When the count is greater
        than zero the user is logged in and upon first entry a value will be inserted into the
        Log In Time field. When the value of Count reaches zero again a value will be inserted into
        the Log Out Time field.

        \subsection{Security}

        {\bf Access Control List Table}\\
        \\
        \\
%        \includegraphics{Tables/ACL.png}\\
        \\
        \\
        The ACL table which is short for Access Control List Table manages which user types have access
        to different levels of program use. All this table contains is a Name of a user type and a description
        of their functionality. Currently we foresee only four user types and will likely not have an
        interface to manage this table.\\
        \\
        \\
        {\bf Key Table}\\
        \\
        \\
%        \includegraphics{Tables/KeyTable.png}\\
        \\
        \\
        The Key Table is used to manage the keys used to unencrypt the data stored in the database. The
        entry in this table will reference a valid user in the Employee Table and the text field will
        contain encrypted data that the user's password was used to encrypt. When the password is used
        to decrypt this data the user can obtain the keys used to encrypt global database data. A few
        example fields that are going to be stored as encrypted data are Credit Card Numbers and Expiration
        Date pairs as well as Social Security Numbers. This will add a level of database security that
        will prevent or deter malicious users from stealing valuable information from the database.\\
        \\

    \newpage

    \section{YardSale Client Application Design}
    The YardSale client is a modular structure designed to
    be flexible enough to accommodate any style of business.
    Each screen uses a bottom-oriented toolbar containing access
    to the on-screen calculator and keyboard, as well as the
    current time, an UNDO button, and a backwards navigation
    button.\\
    \subsection{Main Menu}
    \includegraphics{ys_main_screener.png}\\

    The Main Menu is designed to allow users to quickly
    access any part of the YardSale client.  Buttons are available
    to users depending on access level.  The screen shown above
    displays an administrative user's access, as all options are
    currently available. The bottom toolbar does not contain the
    backwards navigation button, as this is the top-level screen.
    The buttons are enlarged to support touchscreen access.\\

    \subsection{Sales Screen}
    \includegraphics{ys_sales_screener.png}\\

    The Sales Screen is designed to provide much information and functionality, in
    an easy-access and user-friendly manner.  It displays
    information about the current customer, which may be edited to
    make any corrections or updates.  Below this is an inventory item
    list, which expands into a tree form to list similar
    products.  This section may be used in the case of a barcode
    scanner malfunction, or to give information about items similar
    to the ones being purchased.  To the right of the screen is the
    transaction shopping cart, which displays names and prices of
    items currently entered into the system to be purchased during this
    transaction.  When items are present in this list, a running
    total (including tax) is displayed at the bottom of the list.
    The trash can icon is used to remove unwanted items from the
    list.\\

    \subsection{Inventory Screen}
    \includegraphics{ys_inv_screener.png}\\

    The Inventory Screen can display information pertaining to all
    inventory items currently stored in the database.  There are
    fields available for every possible value of an
    inventory item, which may be edited and saved.  In addition to these
    characteristics, the Inventory Screen has many other functions which it can perform.
    It can be used to add items to the inventory by filling in the
    appropriate fields and clicking the 'New Item' button.  The
    Search function is used by entering the desired information
    into the appropriate fields and clicking the 'Search' button.
    This will cause all inventory items present in the database
    that satisfy the criteria to be displayed in the window at the
    bottom of the screen.  For error checking purposes, a 'Save'
    and a 'Cancel' button are used when exiting this screen.  This
    screen should only be available to manager-access-level users
    and higher.\\

    \subsection{Employee Management Screen}
    \includegraphics{ys_employee_screener.png}\\

    The Employee Management Screen is designed to store and
    display information pertaining to all employees currently in the
    system.  There are fields available for every possible value
    of an employee item in the database.  Information is displayed
    by entering the desired employee's ID number and pressing
    enter.  This screen should be carefully guarded, as sensitive
    information is displayed.  It should only be available to
    manager-access-level users and higher.\\

    \newpage

    \subsection{On-screen Calculator}
    \includegraphics{ys_calc_screener.png}\\

    The On-Screen Calculator is designed to provide the user with
    an easy-access, user-friendly calculator and number pad.  It
    is available to all users from each screen in the system, via the tool bar
    at the bottom of the screen.  It provides the functionality of
    a basic calculator (add and subtract, multiply and divide to
    be added at a later date).  Numbers entered and the results
    are displayed in the box at the top of the calculator.\\

\chapter{Implementation}

    \section{Database Level Implementation}


        \subsection{Customer Management}
        {\tt\small
        DROP TABLE IF EXISTS Customer\_Table;\\

        CREATE TABLE Customer\_Table(
        \begin{list}{}
            \item{CUST\_Account\_Number INT AUTO\_INCREMENT NOT NULL,}
            \item{CUST\_First\_Name varchar(25),}
            \item{CUST\_Middle\_Name varchar(25),}
            \item{CUST\_Last\_Name varchar(50),}
            \item{CUST\_Address TEXT,}
            \item{CUST\_Phone varchar(20),}
            \item{CUST\_City varchar(50),}
            \item{CUST\_Zip varchar(12),}
            \item{CUST\_Credit\_Card\_Number varchar(20),}
            \item{CUST\_CC\_Exp\_Date varchar(6),}
            \item{CUST\_Name\_On\_CC TEXT,}
            \item{CUST\_Signature TEXT,}
            \item{CUST\_Photo TEXT,}
            \item{Primary Key(CUST\_Account\_Number)}
        \end{list}
        )type=InnoDB\\
        }

        \subsection{Inventory Management}
        {\tt\small
        DROP TABLE IF EXISTS Inventory\_Table;\\

        CREATE TABLE Inventory\_Table(
        \begin{list}{}
            \item{INV\_SKU\_Number                  varchar(10),}
            \item{INV\_Bar\_Code\_Number             varchar(30),}
            \item{INV\_Item\_Description            TEXT,}
            \item{INV\_Item\_Department             varchar(30),}
            \item{INV\_Quantity\_On\_Hand            INT,}
            \item{INV\_Quantity\_On\_Order           INT,}
            \item{INV\_Reorder\_Level               INT,}
            \item{INV\_Reorder\_Quantity            INT,}
            \item{INV\_Item\_Type                   varchar(20),}
            \item{INV\_REF\_TAX\_Tax\_Type            INT NOT NULL,}
            \item{INV\_REF\_VND\_Vendor\_ID           INT NOT NULL,}
            \item{INV\_Retail\_Price                DECIMAL(7,2),}
            \item{INV\_Wholesale\_Price             DECIMAL(7,2),}
            \item{INV\_Bulk\_Price                  TEXT,}
            \item{INV\_Date\_Last\_Received          DATETIME,}
            \item{INV\_Weight\_Pounds               FLOAT,}
            \item{INV\_Oversized\_Flag              enum('T','F'),}
            \item{INV\_Ship\_By\_Freight             enum('T','F'),}
            \item{INV\_Comment                     TEXT,}
            \item{Primary Key (INV\_SKU\_Number),}
            \item{UNIQUE INDEX (INV\_Bar\_Code\_Number),}
            \item{INDEX tax\_id (INV\_REF\_TAX\_Tax\_Type),}
            \item{INDEX vnd\_id (INV\_REF\_VND\_Vendor\_ID),}
            \item{FOREIGN KEY (INV\_REF\_TAX\_Tax\_Type)}
            \item{REFERENCES Tax\_Table(TAX\_ID),}
            \item{FOREIGN KEY (INV\_REF\_VND\_Vendor\_ID)}
            \item{REFERENCES Vendor\_Table(VND\_ID)}
        \end{list}
        ) type=InnoDB\\
        }

        {\tt\small
        DROP TABLE IF EXISTS Tax\_Table;\\

        CREATE TABLE Tax\_Table(
        \begin{list}{}
            \item{TAX\_ID          INT AUTO\_INCREMENT NOT NULL,}
            \item{TAX\_Name        varchar(20),}
            \item{TAX\_Percent     FLOAT,}
            \item{Primary Key (TAX\_ID)}
        \end{list}
        )type=InnoDB\\
        }

        \subsection{Transaction Handling}
        {\tt\small
        DROP TABLE IF EXISTS Transaction\_Log\_Table;\\

        CREATE TABLE Transaction\_Log\_Table(
        \begin{list}{}
            \item{TRANS\_REF\_EMP\_ID\_Number                 INT,}
            \item{TRANS\_REF\_INV\_SKU\_Number                varchar(10),}
            \item{TRANS\_REF\_CUST\_Account\_Number           INT,}
            \item{TRANS\_Sale\_Price                        DECIMAL(10,2),}
            \item{TRANS\_ID                                INT NOT NULL,}
            \item{TRANS\_Quantity                          INT,}
            \item{TRANS\_Comment                           TEXT,}
            \item{Primary Key(TRANS\_REF\_EMP\_ID\_Number,}
            \begin{list}{}
                \item{TRANS\_REF\_INV\_SKU\_Number,}
                \item{TRANS\_REF\_CUST\_Account\_Number,}
                \item{TRANS\_Sale\_Price, TRANS\_ID,}
                \item{TRANS\_Quantity),}
            \end{list}
            \item{INDEX trans\_id (TRANS\_ID),}
            \item{INDEX emp\_id (TRANS\_REF\_EMP\_ID\_Number),}
            \item{INDEX sku\_num (TRANS\_REF\_INV\_SKU\_Number),}
            \item{INDEX cust\_acct (TRANS\_REF\_CUST\_Account\_Number),}
            \item{FOREIGN KEY (TRANS\_REF\_EMP\_ID\_Number) REFERENCES Employee\_Table(EMP\_ID\_Number),}
            \item{FOREIGN KEY (TRANS\_REF\_INV\_SKU\_Number) REFERENCES Inventory\_Table(INV\_SKU\_Number),}
            \item{FOREIGN KEY (TRANS\_REF\_CUST\_Account\_Number) REFERENCES Customer\_Table(CUST\_Account\_Number)}
        \end{list}
        )type=InnoDB\\
        }

        {\tt\small
        DROP TABLE IF EXISTS Package\_Table;\\

        CREATE TABLE Package\_Table(
        \begin{list}{}
            \item{PKG\_ID\_Number                           INT AUTO\_INCREMENT NOT NULL,}
            \item{PKG\_REF\_TRANS\_ID                        INT,}
            \item{PKG\_REF\_CUST\_Account\_Number             INT,}
            \item{PKG\_REF\_CRR\_ID\_Number                   INT,}
            \item{PKG\_Tracking\_Number varchar(50),}
            \item{PKG\_REF\_SHP\_Shipping\_Type varchar(30),}
            \item{Primary Key(PKG\_ID\_Number),}
            \item{INDEX trans\_id (PKG\_REF\_TRANS\_ID),}
            \item{INDEX cust\_acct (PKG\_REF\_CUST\_Account\_Number),}
            \item{INDEX crr\_id (PKG\_REF\_CRR\_ID\_number),}
            \item{INDEX shp\_type (PKG\_REF\_SHP\_Shipping\_Type),}
            \item{FOREIGN KEY (PKG\_REF\_TRANS\_ID) REFERENCES Transaction\_Log\_Table( TRANS\_ID ),}
            \item{FOREIGN KEY (PKG\_REF\_CUST\_Account\_Number) REFERENCES Customer\_Table( CUST\_Account\_Number ),}
            \item{FOREIGN KEY (PKG\_REF\_CRR\_ID\_Number) REFERENCES Carrier\_Table( CRR\_ID ),}
            \item{FOREIGN KEY (PKG\_REF\_SHP\_Shipping\_Type) REFERENCES Shipping\_Table( SHP\_Type )}
        \end{list}
        ) type=InnoDB\\
        }

        {\tt\small
        DROP TABLE IF EXISTS Carrier\_Table;\\

        CREATE TABLE Carrier\_Table(
        \begin{list}{}
            \item{CRR\_ID                          INT AUTO\_INCREMENT NOT NULL,}
            \item{CRR\_Name                        varchar(50),}
            \item{CRR\_Pickup\_Location             TEXT,}
            \item{CRR\_Phone\_Number                varchar(20),}
            \item{Primary Key (CRR\_ID)}
        \end{list}
        )type=InnoDB\\
        }

        {\tt\small
        DROP TABLE IF EXISTS Shipping\_Table;\\

        CREATE TABLE Shipping\_Table(
        \begin{list}{}
            \item{SHP\_Type                varchar(30),}
            \item{SHP\_REF\_CRR\_ID          INT,}
            \item{SHP\_Cost                DECIMAL(7,2),}
            \item{Primary Key(SHP\_Type, SHP\_REF\_CRR\_ID, SHP\_Cost),}
            \item{INDEX crr\_id (SHP\_REF\_CRR\_ID ),}
            \item{INDEX shp\_type (SHP\_Type),}
            \item{FOREIGN KEY (SHP\_REF\_CRR\_ID) REFERENCES Carrier\_Table( CRR\_ID )}
        \end{list}
        )type=InnoDB\\
        }

        \subsection{Employee Management}
        {\tt\small
        DROP TABLE IF EXISTS Employee\_Table;\\

        CREATE TABLE Employee\_Table(
        \begin{list}{}
            \item{EMP\_Social\_Security\_Number             varchar(13) NOT NULL,}
            \item{EMP\_ID\_Number                           INT NOT NULL,}
            \item{EMP\_First\_Name                          varchar(25),}
            \item{EMP\_Middle\_Name                         varchar(25),}
            \item{EMP\_Last\_Name                           varchar(50),}
            \item{EMP\_Address                              TEXT,}
            \item{EMP\_Phone\_Number                        varchar(20),}
            \item{EMP\_City                                 varchar(50),}
            \item{EMP\_Zip                                  varchar(12),}
            \item{EMP\_Picture                              TEXT,}
            \item{EMP\_Signature                            TEXT,}
            \item{EMP\_REF\_ACL\_Type                       varchar(30),}
            \item{\#EMP\_REF\_CUST\_Account\_Number            INT,}
            \item{Primary Key (EMP\_Social\_Security\_Number),}
            \item{UNIQUE INDEX(EMP\_ID\_Number),}
            \item{INDEX acl\_type (EMP\_REF\_ACL\_Type),}
            \item{\#INDEX acct\_number (EMP\_REF\_CUST\_Account\_Number),}
            \item{FOREIGN KEY (EMP\_REF\_ACL\_Type) REFERENCES ACL\_Table(ACL\_Type),}
            \item{\#FOREIGN KEY (EMP\_REF\_CUST\_Account\_Number) REFERENCES Customer\_Table(CUST\_Account\_Number)}
        \end{list}
        ) type=InnoDB\\
        }

        {\tt\small
        DROP TABLE IF EXISTS Key\_Table;\\

        CREATE TABLE Key\_Table(
        \begin{list}{}
            \item{KEY\_ID                          INT AUTO\_INCREMENT,}
            \item{KEY\_REF\_EMP\_ID\_Number        INT,}
            \item{KEY\_Keys                        TEXT,}
            \item{PRIMARY KEY ( KEY\_ID ),}
            \item{INDEX ( KEY\_REF\_EMP\_ID\_Number ),}
            \item{FOREIGN KEY (KEY\_REF\_EMP\_ID\_Number) REFERENCES Employee\_Table( EMP\_ID\_Number )}
        \end{list}
        )type=InnoDB\\
        }

        {\tt\small
        DROP TABLE IF EXISTS ACL\_Table;\\

        CREATE TABLE ACL\_Table(
        \begin{list}{}
            \item{ACL\_Type varchar(30),}
            \item{ACL\_Description TEXT,}
            \item{Primary Key (ACL\_Type)}
        \end{list}
        ) type=InnoDB\\
        }

    \section{Application Level Implementation}

        \subsection{YardCalc}
        YardCalc is a generic on-screen calculator the user employs to
        enter prices.  This calculator uses a stack-based method of
        storing numbers and operators.\\
        {\sl SEE YARDSALE IMPLEMENTATION MANUAL SECTION 7.3}

        \subsection{YardDatabase}
        YardDatabase is the main database backend which does all
        translation from OO calls to SQL/ODBC.\\
        {\sl SEE YARDSALE IMPLEMENTATION MANUAL SECTION 7.6}

        \subsection{YardDBType}
        YardBDType is the abstract base class for all database
        objects.  All database types are assignable and contain a
        ToString() method to format the DB type to text.\\
        {\sl SEE YARDSALE IMPLEMENTATION MANUAL SECTION 7.8}

            \subsubsection{YardEmployeeType}
            YardEmployeeType is a subclass of YardDBType, which
            represents an employee record and contains functions for
            all possible items.\\
            {\sl SEE YARDSALE IMPLEMENTATION MANUAL SECTION 7.11}

            \subsubsection{YardInvType}
            YardInvType is a subclass of YardDBType, which
            represents an employee record and contains functions for
            all possible items.\\
            {\sl SEE YARDSALE IMPLEMENTATION MANUAL SECTION 7.15}
            \begin{list}{}
                \item {\bf YardInvType::BulkPricing}\\BulkPricing is a
                C++ structure that associates a quantity with a
                percentage for bulk pricing.\\
                {\sl SEE YARDSALE IMPLEMENTATION MANUAL SECTION 7.16}
            \end{list}

        \subsection{YardEmployee}
        YardEmployee is the employee management screen.  Depending on
        access level, users may insert/modify employee information via
        this screen.\\
        {\sl SEE YARDSALE IMPLEMENTATION MANUAL SECTION 7.10}

        \subsection{YardInventory}
        YardInventory is the inventory management screen, which allows
        searching.  Depending on access level, users may add inventory
        via the "New Item" button on this screen.\\
        {\sl SEE YARDSALE IMPLEMENTATION MANUAL SECTION 7.14}

        \subsection{YardException}
        YardException is the exception class from Crypto++.\\
        {\sl SEE YARDSALE IMPLEMENTATION MANUAL SECTION 7.12}

        \subsection{YardLog}
        YardLog is the logging widget, based on wxListCtrl and wxLog.
        This widget resets the default logging system and redirects
        all output to itself.  Different icons represent what type of
        log message is being displayed.\\
        {\sl SEE YARDSALE IMPLEMENTATION MANUAL SECTION 7.17}

        \subsection{YardLogin}
        YardLogin is the customized login screen.  The user will be
        asked for a username and password.  Also, a quick select icon
        will allow the user to rapidly select his/her name.\\
        {\sl SEE YARDSALE IMPLEMENTATION MANUAL SECTION 7.18}

        \subsection{YardMain}
        YardMain is the main menu screen, which displays graphical
        buttons for accessing each part of the system.\\
        {\sl SEE YARDSALE IMPLEMENTATION MANUAL SECTION 7.19}

        \subsection{YardSale}
        YardSale is the main application object, which returns a
        reference to a YardDatabase object.\\
        {\sl SEE YARDSALE IMPLEMENTATION MANUAL SECTION 7.20}

        \subsection{YardSaleScreen}
        YardSaleScreen is main sale screen, which contains the current
        transaction information and an interface to add new items to
        the transaction.  The payment screen can be accessed from
        YardSaleScreen.\\
        {\sl SEE YARDSALE IMPLEMENTATION MANUAL SECTION 7.21}

        \subsection{YardSplash}
        YardSplash is an eye-candy, startup splash screen that shows a
        progress bar.\\
        {\sl SEE YARDSALE IMPLEMENTATION MANUAL SECTION 7.22}

\chapter{Preliminary Test Plan}
The YardSale system is tested constantly throughout the
development process.  These tests are administered through two
different methods: passive (automatic) testing, and active (user)
testing.

    \section{Passive Testing}

    The passive testing of YardSale is accomplished by our auto build system.  The auto build system pulls the latest source
    code from the repository and compiles it with the most strict settings.  If the code fails the system reports the error
    on our website.  The auto build system will also compile the code on six other architechures including Amd64, Sparc, Mac OSX,
    Linux, and Microsoft Windows.  The extensive compilation of the source code ensures that no platform dependant source code
    violates the portability of the project.

    \section{Active Testing}
    There are several methods which are utilized by our
    development team to manually test the YardSale system
    throughout the development process.
        \subsection{Component Testing}
	Each component that utilizes the database is tested by creating "dummy" data which is loaded directly into the database backend.  The each database interface screen is tested by logging into YardSale client and assuring that each dataset is loaded correctly.  Inserts into the database are verified through a third-party database interface tool.
        \subsection{Testing Mains}
        Each non-gui object in the YardSale system has an integrated main loop which allows the program to build test versions of
        each of the objects.  These test objects then execute a series of tests to ensure that the object is working exactly to specification.  These tests verify that all components are able to store, copy, and construct data.  The mains also run simple database interface tests to verify that all of the database tables can re returned correctly.
        If any of these tests fail the system will report the exact location of the failure by the use of function macros.

        \subsection{Warm-Body Testing}
        Each GUI screen is tested by our virtual employees for both functionality and usability.  The employees report back to the developers with any
        bugs they find through our bug tracking system.  We also have weekly usability conferences where we discuss the merits of a particular interface.

        \subsection{Program Flow Tracking}
        YardSale is designed to report back a large amount of valuable debug information to the developer and to keep the non-technical user
        informed about the status of the program.  YardSale uses exceptions to ensure that single functions cannot violate the integrity of the system
        on a global level.  In addition, all database interfaces have special logging information that reports the exact status of all database calls; this
        allows the developers to get instant access to valuable SQL and connection results.

\end{document}
