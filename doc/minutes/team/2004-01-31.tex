\documentclass{article}

\begin{document}

\begin{flushleft}
{\bf Team Meeting}\\
{\bf January 31, 2004}\\
{\bf 4:00pm}\\
\end{flushleft}

\begin{flushleft}
Attendance:\\
Jesse Lovelace\\
Adam Parrish\\
Mike Swigon\\
Jay Johnston\\
Cameron Watts\\
\end{flushleft}

\section{Developmental Instruments}
\subsection{sourceforge.net}
an online repository for open source projects.  includes bug
tracking, CVS, forums, mailing lists...

\begin{itemize}
    \item explain use of sourceforge.net to all team members not familiar with the repository
    \item each team member needs to create an account with sourceforge.net
    \item each team member needs to subscribe to appropriate mailing lists
\end{itemize}

\subsection{\LaTeX}
a high-quality typesetting system, with features designed for the
production of technical and scientific documentation. \\{\bf EACH
OF THE FOLLOWING COMMANDS CAN BE FOUND IN THE \LaTeX EXAMPLE FILE
CREATED BY ADAM}

\begin{description}
    \item[sectioning] organizing information into sections and subsections
    \item[emphasizing text] showing text in italics and boldface
    \item[math mode] showing formulas and special characters in math mode
    \item[lists] showing information in list forms
    \item[images and diagrams] inserting and displaying external images and diagrams in \LaTeX files
\end{description}

\section{Coding Standards}
jesse has prepared a presentation for the team explaining the
standards for coding modules pertaining to the openPOS system to
be created by the team. \\{\bf OUTLINED BY THE FOLLOWING:}

\begin{description}
    \item[headers] will follow the standards defined by the JavaDoc commenting style
    \item[spacing] lines following any control structure or declaration should be preceded by 4 additional spaces from the location of the previous line.  we will NOT be using TABs interchangeably with 4 spaces, as TABs are application dependant and spaces are not.
    \item[variable names] will be started with a lowercase letter, and each subsequent word begun with an uppercase letter; no spacing between words; EXAMPLE '{\it variableName}'
    \item[member variables] will be denoted by preceding the variable name with 'm\_'; EXAMPLE 'm\_{\it variableName}'
\end{description}

\section{Goal/Due Dates}
\subsection{Long Term}

\begin{description}
    \item[2004-03-05] FEATURE FREEZE - deadline for adding functional features to the design of the POS
    \item[2004-04-17] CODE FREEZE - deadline for beginning the coding of features; those not already implemented will be removed; testing only beyond this point
    \item[2004-04-27] POSTERS and PIES - final deadline; presentation of project to CSC department
\end{description}

\subsection{Short Term}
\begin{description}
    \item[2004-02-04] DB structure - adam and mike to complete the preliminary design of the DB tables
    \item[2004-02-11]
    \begin{itemize}
        \item GUI layout - jesse to complete the preliminary design of the user interface
        \item SQL queries - adam and mike to complete preliminary SQL commands necessary for populating the DB
        \item research - entire team to complete reserch on transaction processing and report
    \end{itemize}
    \item[2004-02-15] POS research - cameron to conduct research on POS systems currently in use by retail establishments; PICTURES AND USER FEEDBACK NEEDED.
    \item[2004-02-29] interfacing - john and jay to research interfacing with barcode and magnetic card scanners
\end{description}



\end{document}
